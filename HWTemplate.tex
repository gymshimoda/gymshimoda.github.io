% --------------------------------------------------------------
% This is all preamble stuff that you don't have to worry about.
% Head down to where it says "Start here"
% --------------------------------------------------------------
 
\documentclass[12 pt]{article}
\usepackage{amsmath, amssymb, amsthm}

\theoremstyle{definition}
% May want theorems numbered by chapter
\newtheorem*{thm}{Theorem}
\newtheorem*{prob}{Problem}
\newtheorem*{ex}{Exercise}

\newcommand{\R}{\mathbb{R}}
\newcommand{\N}{\mathbb{N}}
\newcommand{\Z}{\mathbb{Z}}

\newcommand\m[1]{\begin{pmatrix}#1\end{pmatrix}}
\newcommand\x[1]{\textbf{Grade - $\times$ - resubmit by {#1}}}
\newcommand\c{\textbf{Grade - $\checkmark$}
 
\begin{document}
 
% --------------------------------------------------------------
%                         Start here
% --------------------------------------------------------------
 
\title{Weekly Homework X}%replace X with the appropriate number
\author{Tony Stark\\ %replace with your name
Linear Algebra\\
Collaborators: } %if necessary, include the names of everyone you worked with on this.
 
\maketitle
 
\begin{prob}{1}\\ %You can use theorem, exercise, or problem.  Modify 1 to be whatever number you are proving
Delete this text and write theorem statement here.
\end{prob}


\begin{proof}
Blah, blah, blah.  Here is an example of the \texttt{align} environment:
%Note 1: The * tells LaTeX not to number the lines.  If you remove the *, be sure to remove it below, too.
%Note 2: Inside the align environment, you do not want to use $-signs.  The reason for this is that this is already a math environment. This is why we have to include \text{} around any text inside the align environment.
\begin{align*}
\sum_{i=1}^{k+1}i & = \left(\sum_{i=1}^{k}i\right) +(k+1)\\ 
& = \frac{k(k+1)}{2}+k+1 & (\text{by inductive hypothesis})\\
& = \frac{k(k+1)+2(k+1)}{2}\\
& = \frac{(k+1)(k+2)}{2}\\
& = \frac{(k+1)((k+1)+1)}{2}.
\end{align*}
\end{proof}
 
\begin{thm}{2}\\
Let $n\in \Z$.  Then yada yada.
\end{thm}
 
\begin{proof}
Blah, blah, blah.  I'm so smart.
\end{proof}

\begin{ex}{3}\\
Show that a matrix does stuff. Oh yeah.
\end{ex}

\begin{proof}
Look, here's a matrix.
$$A = \m{1&0&0\\0&1&0\\0&0&1}$$

There's a vector too!

$$\vec{v} = \m{1\\2\\3}$$
\end{proof}

% --------------------------------------------------------------
%     You don't have to mess with anything below this line.
% --------------------------------------------------------------
 
\end{document}
